\documentclass{article}
\usepackage{amsmath}
\usepackage{xcolor}
\begin{document}
    \tableofcontents
    \newpage
    \section{Supervised Machine Learning: Regression and Classification}
    \subsection{Week 1 linear Regression with One Variable}
	\subsubsection{Learning Objectives}
    \begin{itemize}
	    \item Define machine learning
        \item Define supervised learning
        \item Define unsupervised learning
        \item Write and run Python code in Jupyter Notebooks
        \item Define a regression model
        \item Implement and visualize a cost function
        \item Implement gradient descent
        \item Optimize a regression model using gradient descent
    \end{itemize}  
    \subsubsection*{Some notations}
    \begin{itemize}
	    \item In the training set, there are features ($x$) and targets ($y$)
        \item $x$ is the input variable (feature)
        \item $y$ is the output variable (target)
        \item $m = \text{number of training examples}$
        \item $(x,y) = \text{single training example}$
        \item \textcolor{red}{$(x^{(i)},y^{(i)}) = i^{th}\text{ training example } (1^{st},2^{nd},3^{rd}...)$ }
        \item \textcolor{red}{$\hat{y}$ means prediction(estimated y)}
        \item $f$ is the function (model)(hypothesis)
        \item $f_{w,b}(x)=wx+b$
        \item Parameters: $w,b$
        \item \textcolor{red}{Squared error cost function: $J(w,b)=\frac{1}{2m}\sum_{i=1}^{m}(\hat{y}^{(i)}-y^{(i)})^2$}
        \item Find $w,b$ : $\hat{y}^{(i)}\text{ is close to }y^{(i)}\text{ for all }(x^{(i)},y^{(i)})$
    \end{itemize} 
    \[ J(w,b)=\frac{1}{2m}\sum_{i=1}^{m}(f_{w,b}(x^{(i)})-y^{(i)})^2 \]
    \begin{itemize}
        \item Goal: $minimizeJ(w,b)$
    \end{itemize}
    \subsubsection{Gradient Descent}
    repeat until convergence\{
    \[ w=w-\alpha\frac{\partial}{\partial w}J(w,b) \]
    \[ b=b-\alpha\frac{\partial}{\partial b}J(w,b) \]
    \} \newline
    Simultaneously update w and b \newline
    $\alpha\text{ is the learning rate(usually between 0 and 1, maybe 0.01)}$ \newline
    $\frac{\partial}{\partial w}J(w,b)\text{ and }\frac{\partial}{\partial b}J(w,b)\text{ are derivative terms}$ \newline
    Near a local minimum,
    \begin{itemize}
        \item Derivative becomes smaller
        \item Update steps become smaller
    \end{itemize}
    \subsubsection{Gradient descent for linear regression}
    \[\frac{\partial}{\partial w}J(w,b) = \frac{1}{m}\sum_{i=1}^{m}(f_{w,b}(x)^{(i)}-y^{(i)})x^{(i)}\]
    \[\frac{\partial}{\partial b}J(w,b) = \frac{1}{m}\sum_{i=1}^{m}(f_{w,b}(x)^{(i)}-y^{(i)})\]
    repeat until convergence\{
    \[w = w - \alpha\frac{1}{m}\sum_{i=1}^{m}(f_{w,b}(x)^{(i)}-y^{(i)})x^{(i)}\]
    \[b = b - \alpha\frac{1}{m}\sum_{i=1}^{m}(f_{w,b}(x)^{(i)}-y^{(i)})\]
    \} 
    \subsubsection{Lab notes}
    \paragraph{}
    Deep copy is a process in which the copying process occurs recursively. It means first constructing a new collection object and then recursively populating it with copies of the child objects found in the original. In case of deep copy, a copy of object is copied in other object. It means that any changes made to a copy of object do not reflect in the original object. In python, this is implemented using “deepcopy()” function. \\
    0.3e : example: 1.949e+02 (Three decimal places are reserved before e) \\
    8.4f : example: 199.9929 (Including the decimal point, there are eight decimal places, and four decimal places are reserved after the decimal point) 
    \subsection{Week 2 Regression with multiple input variables}
    \subsubsection{Learning Objectives}
    \begin{itemize}
        \item Use vectorization to implement multiple linear regression
        \item Use feature scaling, feature engineering, and polynomial regression to improve model training
        \item Implement linear regression in code
    \end{itemize}
    \subsubsection{Multiple Linear Regression}





    \newpage
    $ 2^{2} + 2^{2} = 8$ and $ 2 \times 2 = 4$

    \[ \cos^2 \theta + \sin^2 \theta = 1\]
    
    The union of two sets A and B is denoted as 
    $ A \cup B = \{ x \in A \ \text{or} \ x \in B \} $
    
    We are learning fractions $ \frac{a}{\frac{b}{c}} \times \frac{\frac{d}{e}}{f} \geq 1$ which is good.

    \[ \frac{a}{\frac{b}{c}} \times \frac{\frac{d}{e}}{f} \geq 1\]

    \[ \Bigg\{ \bigg(\frac{a}{b}\bigg) + \bigg(\frac{c}{d}\bigg)\Bigg\} \]

    \[ \sum_{i=a}^{b} g(i) = 0, \text{for } b < a \]

    \[ \sum_{i=1}^{n} i = \frac{n(n+1)}{2} \]

    \section{More in Mathematics}

    $\int_{2}^{4}$

    $lim_{2 \to 4}$

    \[ \int_{0}^{\infty} f(x)dx \]

    \[ \lim_{x \to c} f(x)=L \]

    $
    \begin{bmatrix}
        1 & 2 \\
        3 & 4 \\
    \end{bmatrix}
    +
    \begin{bmatrix}
        5 & 6 \\
        7 & 8 \\
    \end{bmatrix}
    =
    \begin{bmatrix}
        6 & 8 \\
        10 & 12 \\
    \end{bmatrix}
    $

    \section{Equations}
    \begin{equation}
        3x + 5y = 2
    \end{equation}
    
    \begin{equation}
        5x + 8y = 3
    \end{equation}

    \begin{equation}
        x^{2} - y^{2} = (x+y)(x-y)
    \end{equation}

    \begin{align}
        3x - 6 &= 9 \\
        3x &= 9 + 6 \nonumber \\
        x &= \frac{9+6}{3} \nonumber \\
        x &= 5 \nonumber
    \end{align}
\end{document}